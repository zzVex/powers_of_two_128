
\documentclass[12pt]{article}
\usepackage[utf8]{inputenc}
\usepackage[T1]{fontenc}
\usepackage[russian]{babel}
\usepackage{lmodern}
\usepackage{amsmath,amssymb,amsthm}
\usepackage[a4paper,margin=1in]{geometry}
\usepackage{enumitem}
\usepackage{hyperref}

\title{Степени двойки с цифрами из $\{1,2,8\}$: комбинаторное доказательство конечности}
\author{Горюшкин С.В.}
\date{\today}

\theoremstyle{plain}
\newtheorem{theorem}{Теорема}
\newtheorem{lemma}{Лемма}
\newtheorem{corollary}{Следствие}
\theoremstyle{remark}
\newtheorem*{remark}{Замечание}
\newtheorem*{example}{Пример}

\begin{document}
\maketitle

\begin{abstract}
Мы даём полностью элементарное доказательство того, что единственные степени двойки, чья десятичная запись использует только цифры из $\{1,2,8\}$, это $2,8,128$. Ключевые идеи: (i) фазовое условие $2^n\equiv2\pmod6$ для нечётных $n$; (ii) инвариант состава цифр $a\equiv b\pmod3$ (число едениц и число цифр из $\{2,8\}$ в старших разрядах равны по модулю $3$); (iii) явный разбор допустимых двух- и трёхзначных хвостов; (iv) запрет длины $4$ через сумму цифр и делимость на $16$; (v) запрет длин $\ge5$ с помощью инварианта.
\end{abstract}

\section{Постановка}
Назовём число \emph{валидным}, если его десятичная запись содержит только цифры из $\{1,2,8\}$. Нас интересуют валидные степени двойки $N=2^n$. Очевидно, $2$ и $8$ валидны, а также $128=2^7$ валидно. Покажем, что других нет.

\section{Фаза по модулю 6 и структура разрядов}
Отметим две элементарные наблюдения.
\begin{itemize}[nosep]
\item Для нечётных $n$ имеем $2^n\equiv 2\pmod6$ (\emph{фаза $2$}); для чётных $n$ остаток $4$, а последняя цифра $4$ или $6$, что невалидно. Значит, всегда $n$ нечётно, а $N\equiv2\pmod6$.
\item В $\Bbb Z/6\Bbb Z$ справедливо $10\equiv4$ и, следовательно, $10^j\equiv4$ для всех $j\ge1$.
\end{itemize}

\begin{lemma}[Инвариант состава по модулю 6]\label{lem:inv}
Пусть валидное $N$ имеет последнюю цифру $u\in\{2,8\}$, а выше единиц стоят $a$ штук цифры $1$ и $b$ штук цифр из $\{2,8\}$. Тогда
\begin{equation}\label{eq:ab}
2a+b\equiv0\pmod3,\qquad a+2b\equiv0\pmod3,\qquad\text{в частности }~ a\equiv b\pmod3.
\end{equation}
\end{lemma}

\begin{proof}
Работаем в $\Bbb Z/6\Bbb Z$. Вклад единиц $u\equiv2$. Каждый старший разряд $1$ даёт вклад $4\cdot1\equiv4$, каждый старший разряд $2$ или $8$ даёт вклад $4\cdot2\equiv8\equiv2$. Итак,
\[
N\equiv 2 + 4a + 2b \pmod6.
\]
Так как $N\equiv2\pmod6$, получаем $4a+2b\equiv0\pmod6$, т. е. $2a+b\equiv0\pmod3$.

С другой стороны, по модулю $3$ сумма цифр равна $N\pmod3$. При нечётном $n$ имеем $2^n\equiv2\pmod3$. Остатки цифр: $1\equiv1$, $2\equiv2$, $8\equiv2$. Следовательно,
\[
(2) + a\cdot1 + b\cdot2 \equiv 2 \pmod3 \quad\Rightarrow\quad a+2b\equiv0\pmod3.
\]
Вычитая две конгруэнции, получаем $a\equiv b\pmod3$.
\end{proof}

\begin{corollary}\label{cor:blocks}
Если $N$ валидно, то добавление слева одной цифры из $\{1,2,8\}$ нарушает $a\equiv b\pmod3$. Следовательно, любое возможное удлинение валидной записи слева должно происходить блоками по три цифры.
\end{corollary}

\begin{example}
Для $N=128$ имеем $a=1$, $b=1$ и $a\equiv b\pmod3$. Добавление слева любой одной цифры (получая $1128$, $2128$ или $8128$) нарушает инвариант (или другие обязательные признаки степени $2$, см. ниже).
\end{example}

\section{Разрешённые хвосты: \texorpdfstring{$\bmod 10$}{mod 10}, \texorpdfstring{$\bmod 100$}{mod 100}, \texorpdfstring{$\bmod 1000$}{mod 1000}}
Здесь мы \emph{явно} выводим допустимые хвосты.

\subsection*{Одна цифра}
Последняя цифра степени $2$ периодична: $2,4,8,6,\ldots$. Для нечётных $n$ остаются $2$ и $8$ — обе валидны.

\subsection*{Две цифры}
Работаем $\bmod 100$. Известно, что порядок $2$ по модулю $25$ равен $20$, а по модулю $4$ — $2$, так что период по модулю $100$ равен $\mathrm{lcm}(20,2)=20$. Достаточно выписать $\{2^n\bmod100:\ n\equiv1,3,\ldots,19\}$ и проверить, у каких остатков обе цифры лежат в $\{1,2,8\}$. Получается ровно три варианта:
\begin{equation}\label{eq:tails2}
\boxed{12,\quad 28,\quad 88.}
\end{equation}
\begin{example}
Например, $2^7=128\equiv 28\ (\bmod 100)$, $2^{19}=524288\equiv 88$, $2^9=512\equiv12$.
\end{example}

\subsection*{Три цифры}
Для $n\ge3$ имеем $2^n\equiv0\pmod8$, значит последние три цифры кратны $8$. Среди всех трёхзначных чисел на алфавите $\{1,2,8\}$ с последней цифрой $2$ или $8$ кратность $8$ оставляет кандидатов
\[
112,\,128,\,288,\,888\quad(\text{остальные, например }212,812,228,828,188,\ldots,~не кратны 8).
\]
Далее используем периодичность по модулю $1000$: порядок $2$ по модулю $125$ равен $100$, а по модулю $8$ — тривиален, так что период по модулю $1000$ равен $100$. Прямая проверка по циклу длины $100$ (или, что то же, решение по китайской теореме об остатках для системы $\bmod 8$ и $\bmod 125$) показывает, что из четырёх кратных $8$ кандидатов реально встречаются \emph{ровно три}:
\begin{equation}\label{eq:tails3}
\boxed{112,\quad 128,\quad 288.}
\end{equation}
Хвоста $888$ в цикле $2^n\bmod1000$ нет.

\begin{example}
\begin{itemize}[nosep]
\item $2^7=128$ имеет хвост $128$;
\item $2^{19}=524288$ имеет хвост $288$;
\item $2^{89}$ имеет хвост $112$ (получается возведением в квадрат от $2^{\! \, \,  89-1}$ с контролем по $\bmod 1000$).
\end{itemize}
\end{example}

\begin{remark}[Как проверить \eqref{eq:tails2}–\eqref{eq:tails3} вручную]
Для двух цифр: вычислите $2^n\bmod25$ для $n=1,3,\ldots,19$ и склейте с правильной последней цифрой по $\bmod4$ (последняя цифра уже фиксирует $\bmod 4$). Для трёх цифр: требование делимости на $8$ резко сокращает список; далее решите систему по CRT для $\bmod 125$ (цикл длины $100$). В обоих случаях это конечная таблица размером не более $20$ и $100$ ячеек соответственно.
\end{remark}

\section{Запрет длины 4}
\begin{lemma}\label{lem:len4}
Никакое четырёхзначное валидное число не является степенью $2$.
\end{lemma}

\begin{proof}
По \eqref{eq:tails3} возможны трёхзначные хвосты только $112,128,288$. Рассмотрим $dXYZ$, где $XYZ\in\{112,128,288\}$ и $d\in\{1,2,8\}$.

\smallskip
\noindent\textbf{(i) Сумма цифр $\bmod 3$.} Для нечётных $n$ имеем $2^n\equiv2\pmod3$, т.~е. сумма цифр $\equiv2\pmod3$. Дадим значения:
\begin{align*}
\mathrm{sum}(d112)&=d+1+1+2\equiv d+1\pmod3,\\
\mathrm{sum}(d128)&=d+1+2+8\equiv d+1\pmod3,\\
\mathrm{sum}(d288)&=d+2+8+8\equiv d+2\pmod3.
\end{align*}
Отсюда сразу запрещены: $d\in\{2,8\}$ для $d112$ и $d128$ (дают $0\bmod3$ вместо $2$), а также все $d288$ (так как $d\equiv1,2,2\pmod3$ и ни одно не даёт $2$). Остаётся единственный кандидат по сумме цифр: $d=1$ в случаях $1112$ и $1128$.

\smallskip
\noindent\textbf{(ii) Делимость на $16$.} Для $n\ge4$ число $2^n$ кратно $16$. Проверим последние четыре цифры кандидатов:
\begin{align*}
1112 &\equiv 8 \pmod{16},\\
1128 &\equiv 8 \pmod{16}.
\end{align*}
Оба не кратны $16$. Следовательно, четырёхзначных валидных степеней $2$ нет.
\end{proof}

\begin{example}
$1128$ выглядит правдоподобно (все цифры допустимы), но не делится на $16$; $2128$, $8128$ нарушают сумму цифр $\bmod3$.
\end{example}

\section{Запрет всех длин $\ge5$}
\begin{lemma}\label{lem:len5}
Валидных степеней $2$ длины $\ge5$ не существует.
\end{lemma}

\begin{proof}
Пусть $N$ валидно. По лемме \ref{lem:inv} инвариант $a\equiv b\pmod3$ должен сохраняться. По следствию \ref{cor:blocks} любое допустимое удлинение/укорочение происходит пакетами по три цифры. Отбрасывая слева по три цифры, мы неизбежно попадём либо в длину $1,2,3$, либо в длину $4$. Случаи $1,2,3$ дают ровно $2,8,128$ (прямой просмотр); длина $4$ невозможна по лемме \ref{lem:len4}. Противоречие.
\end{proof}

\section{Главная теорема и проверяемые примеры}
\begin{theorem}
Единственные валидные степени двойки — это
\[
2^1=2,\qquad 2^3=8,\qquad 2^7=128.
\]
\end{theorem}

\begin{proof}
Нечётность показателя обязательна (фаза $\bmod 6$). По разделу о хвостах единственный реализуемый валидный трёхзначный хвост — $128$ (при $n=7$). По леммам \ref{lem:len4} и \ref{lem:len5} длин $\ge4$ не бывает. Длины $1,2,3$ даются вычислениями: $2^1=2$, $2^3=8$, $2^7=128$.
\end{proof}

\begin{example}
Проверка:
\begin{itemize}[nosep]
\item $2^1=2$ — валидно.
\item $2^3=8$ — валидно.
\item $2^5=32$ — невалидно (цифра $3$).
\item $2^7=128$ — валидно.
\item $2^{19}=524288$ — невалидно (хвост $288$ формально допустим, но есть цифра $5$ слева).
\item Любая попытка $d128$ с $d\in\{1,2,8\}$ — не степень $2$ (см. лемму \ref{lem:len4}).
\end{itemize}
\end{example}

\section*{Приложение A: Как получить хвосты ещё короче}
\paragraph{Две цифры.} Выпишем для нечётных $n$ значения $2^n\bmod100$ (период $20$). Среди них остатки с последней цифрой $2$ или $8$ и десятком из $\{1,2,8\}$ — это ровно $12,28,88$.

\paragraph{Три цифры.} Требуем кратность $8$ (для $n\ge3$) и принадлежность алфавиту. Это сокращает список до $112,128,288,888$. Далее по $\bmod125$ (цикл длины $100$) видно, что $888$ не встречается в ряду $2^n\bmod1000$, а $112,128,288$ встречаются (примерно при $n\equiv 89,7,19\pmod{100}$ соответственно).

\section*{Приложение B: Мини-таблица переносов}
Локальное правило при переходе $2^n\to2^{n+1}$ по правым разрядам: если $d$ цифра, $c\in\{0,1\}$ перенос справа, то новая цифра $e\equiv (2d+c)\bmod10$, новый перенос $c'=\lfloor (2d+c)/10\rfloor$. Требование $e\in\{1,2,8\}$ резко ограничивает варианты. Устойчивый правый мотив — $11\mapsto 22$ без переноса, совместимый с хвостом $112\to128$; следующий шаг уже рождает недопустимую цифру слева от правой пары, что согласуется с запретом длин $\ge4$.



\end{document}
