\documentclass[12pt]{article}
\usepackage[utf8]{inputenc}
\usepackage[T1]{fontenc}
\usepackage[english]{babel}
\usepackage{lmodern}
\usepackage{amsmath,amssymb,amsthm}
\usepackage[a4paper,margin=1in]{geometry}
\usepackage{enumitem}
\usepackage{hyperref}
\usepackage{booktabs}
\usepackage{array}

\title{Powers of Two with Digits from $\{1,2,8\}$: A Combinatorial Proof of Finiteness}
\author{S.\,V. Goryushkin}
\date{\today}

\theoremstyle{plain}
\newtheorem{theorem}{Theorem}
\newtheorem{lemma}{Lemma}
\newtheorem{corollary}{Corollary}
\theoremstyle{remark}
\newtheorem*{remark}{Remark}
\newtheorem*{example}{Example}

\begin{document}
\maketitle

\begin{abstract}
We give a completely elementary proof that the only powers of two whose decimal expansion uses only the digits $\{1,2,8\}$ are $2,8,128$. The key ideas are: (i) the phase condition $2^n\equiv2\pmod6$ for odd $n$; (ii) a digit-composition invariant $a\equiv b\pmod3$ (the numbers of digits equal to $1$ and to $\{2,8\}$ in higher positions are congruent mod $3$); (iii) an explicit derivation of admissible two- and three-digit endings; (iv) exclusion of length $4$ via digit sum modulo $3$ and divisibility by $16$; (v) exclusion of lengths $\ge5$ via the invariant.
\end{abstract}

\section{Setup}
Call a number \emph{valid} if its decimal expansion uses only digits from $\{1,2,8\}$. We are interested in valid powers of two $N=2^n$. Clearly, $2$ and $8$ are valid, and $128=2^7$ is valid as well. We show that there are no others.

\section{Phase modulo 6 and the structure of decimal places}
Two elementary observations:
\begin{itemize}[nosep]
\item For odd $n$ we have $2^n\equiv 2\pmod6$ (\emph{phase $2$}); for even $n$ the residue is $4$, and the last digit is $4$ or $6$, which is invalid. Thus, $n$ must be odd and $N\equiv2\pmod6$.
\item In $\mathbb Z/6\mathbb Z$ one has $10\equiv4$, hence $10^j\equiv4$ for all $j\ge1$.
\end{itemize}

\begin{lemma}[Composition invariant modulo 6]\label{lem:inv}
Let a valid $N$ have last digit $u\in\{2,8\}$, and above the units place let there be $a$ digits equal to $1$ and $b$ digits from $\{2,8\}$. Then
\begin{equation}\label{eq:ab}
2a+b\equiv0\pmod3,\qquad a+2b\equiv0\pmod3,\qquad\text{in particular }~ a\equiv b\pmod3.
\end{equation}
\end{lemma}

\begin{proof}
Work in $\mathbb Z/6\mathbb Z$. The units digit contributes $u\equiv2$. Each higher $1$ contributes $4\cdot1\equiv4$, and each higher $2$ or $8$ contributes $4\cdot2\equiv8\equiv2$. Thus
\[
N\equiv 2 + 4a + 2b \pmod6.
\]
Since $N\equiv2\pmod6$, we get $4a+2b\equiv0\pmod6$, i.e. $2a+b\equiv0\pmod3$.

On the other hand, modulo $3$ the digit sum equals $N\pmod3$. For odd $n$, $2^n\equiv2\pmod3$. With $1\equiv1$, $2\equiv2$, $8\equiv2$, we have
\[
(2)+a\cdot1+b\cdot2\equiv2\pmod3 \Rightarrow a+2b\equiv0\pmod3.
\]
Subtracting gives $a\equiv b\pmod3$.
\end{proof}

\begin{corollary}\label{cor:blocks}
If $N$ is valid, adding a single leftmost digit from $\{1,2,8\}$ breaks $a\equiv b\pmod3$. Hence any possible valid extension on the left must occur in blocks of three digits.
\end{corollary}

\begin{example}
For $N=128$ we have $a=1$, $b=1$ and $a\equiv b\pmod3$. Adding one digit to the left ($1128$, $2128$, $8128$) breaks the invariant or other necessary properties of powers of two.
\end{example}

\section{Admissible endings: \texorpdfstring{$\bmod 10$}{mod 10}, \texorpdfstring{$\bmod 100$}{mod 100}, \texorpdfstring{$\bmod 1000$}{mod 1000}}
We now \emph{explicitly} derive all admissible endings.

\subsection*{One digit}
The last digit of $2^n$ cycles: $2,4,8,6,\ldots$. For odd $n$ only $2$ and $8$ remain — both valid.

\subsection*{Two digits: CRT derivation}
The period of $2^n\bmod100$ is $\mathrm{lcm}(\mathrm{ord}_{25}(2),\mathrm{ord}_{4}(2))=\mathrm{lcm}(20,2)=20$. Consider only odd $n$ (last digit $2$ or $8$). The explicit table of $2^n\bmod100$ for odd $n$ yields exactly three endings whose both digits lie in $\{1,2,8\}$:
\begin{equation}\label{eq:tails2}
\boxed{12,\quad 28,\quad 88.}
\end{equation}

\begin{center}
\renewcommand{\arraystretch}{1.05}
\begin{tabular}{@{}lcccccccccc@{}}
\toprule
$n\bmod 20$ & 1 & 3 & 5 & 7 & 9 & 11 & 13 & 15 & 17 & 19 \\
\midrule
$2^n\bmod 100$ & 2 & 8 & 32 & 28 & 12 & 48 & 92 & 68 & 72 & 88 \\
\bottomrule
\end{tabular}
\end{center}

From this row we see \emph{only} $12,28,88$ as admissible pairs.

\subsection*{Three digits: divisibility by 8 + period modulo $125$}
For $n\ge3$, $2^n\equiv0\pmod8$, so the last three digits are multiples of $8$. Among three-digit numbers over $\{1,2,8\}$ ending in $2$ or $8$, divisibility by $8$ leaves the \emph{candidates}
\[
112,\,128,\,288,\,888.
\]
Next, using $\mathrm{ord}_{125}(2)=100$, the period modulo $1000$ is $100$. Scanning the cycle yields \emph{exactly three} actual endings:
\begin{equation}\label{eq:tails3}
\boxed{112,\quad 128,\quad 288,}
\end{equation}
while $888$ never occurs.

\begin{remark}[Manual check of \eqref{eq:tails2}–\eqref{eq:tails3}]
For two digits: compute $2^n\bmod25$ for $n=1,3,\ldots,19$ and align with $\bmod4$ (the last digit fixes $\bmod4$). For three digits: the $8$-divisibility condition drastically shortens the list; then align with $\bmod125$ (cycle length $100$).
\end{remark}

\section{Mini table of carries (local dynamics)}
When passing from $2^n$ to $2^{n+1}$ right-to-left, if $d$ is a digit and $c\in\{0,1\}$ is the incoming carry, then
\[
e=(2d+c)\bmod10,\qquad c'=\Big\lfloor\frac{2d+c}{10}\Big\rfloor.
\]
Requiring $e\in\{1,2,8\}$ leaves only one locally admissible transition from $\{1,2,8\}$:
\begin{center}
\renewcommand{\arraystretch}{1.1}
\begin{tabular}{@{}cccccc@{}}
\toprule
$d$ & $c$ & $2d+c$ & $e=(2d+c)\bmod10$ & $c'$ & is $e$ valid? \\
\midrule
1 & 0 & 2  & 2 & 0 & \textbf{yes} \\
1 & 1 & 3  & 3 & 0 & no \\
2 & 0 & 4  & 4 & 0 & no \\
2 & 1 & 5  & 5 & 0 & no \\
8 & 0 & 16 & 6 & 1 & no \\
8 & 1 & 17 & 7 & 1 & no \\
\bottomrule
\end{tabular}
\end{center}
Thus the only stable right-hand pattern is \emph{$11\mapsto22$ with no carry}, consistent with the observed ending $112\to128$.

\section{Excluding length 4}
\begin{lemma}\label{lem:len4}
No four-digit valid number is a power of two.
\end{lemma}

\begin{proof}
By \eqref{eq:tails3}, the possible three-digit endings are $112,128,288$. Consider $dXYZ$ with $XYZ\in\{112,128,288\}$ and $d\in\{1,2,8\}$.

\smallskip
\noindent\textbf{(i) Digit sum modulo 3.} For odd $n$, $2^n\equiv2\pmod3$, i.e.\ the digit sum must be $\equiv2\pmod3$. Compute:
\begin{align*}
\mathrm{sum}(d112)&=d+1+1+2\equiv d+1\pmod3,\\
\mathrm{sum}(d128)&=d+1+2+8\equiv d+1\pmod3,\\
\mathrm{sum}(d288)&=d+2+8+8\equiv d+2\pmod3.
\end{align*}
Hence $d\in\{2,8\}$ for $d112$ and $d128$ gives $0\bmod3$ instead of $2$, and all $d288$ are impossible (since $d\equiv1,2,2\pmod3$ never produce $2$). Only $d=1$ remains for $1112$ and $1128$.

\smallskip
\noindent\textbf{(ii) Divisibility by $16$.} For $n\ge4$, $2^n$ is divisible by $16$. Check:
\begin{align*}
1112 &\equiv 8 \pmod{16},\\
1128 &\equiv 8 \pmod{16}.
\end{align*}
Neither is divisible by $16$. Thus no four-digit valid power of two exists.
\end{proof}

\begin{example}
$1128$ looks plausible (all digits allowed) but is not divisible by $16$; $2128$, $8128$ violate the digit-sum mod $3$ condition.
\end{example}

\section{Excluding all lengths $\ge5$}
\begin{lemma}\label{lem:len5}
There are no valid powers of two of length $\ge5$.
\end{lemma}

\begin{proof}
Assume $N$ is valid. By Lemma~\ref{lem:inv}, $a\equiv b\pmod3$ must hold. By Corollary~\ref{cor:blocks}, any valid extension/truncation occurs in blocks of three digits. Removing three digits at a time from the left, we must arrive at a number of length $1,2,3$, or $4$. Lengths $1,2,3$ yield exactly $2,8,128$; length $4$ is impossible by Lemma~\ref{lem:len4}. Contradiction.
\end{proof}

\section{Main theorem and testable examples}
\begin{theorem}
The only valid powers of two (alphabet $\{1,2,8\}$) are
\[
2^1=2,\qquad 2^3=8,\qquad 2^7=128.
\]
\end{theorem}

\begin{proof}
Oddness of the exponent is necessary (phase $\bmod 6$). From the endings section, the only realizable valid three-digit ending is $128$ (at $n=7$). By Lemmas~\ref{lem:len4} and \ref{lem:len5}, no lengths $\ge4$ exist. Lengths $1,2,3$ give $2^1=2$, $2^3=8$, $2^7=128$.
\end{proof}

\begin{example}
Checks:
\begin{itemize}[nosep]
\item $2^1=2$ — valid.
\item $2^3=8$ — valid.
\item $2^5=32$ — invalid (digit $3$).
\item $2^7=128$ — valid.
\item $2^{19}=524288$ — invalid (ending $288$ is admissible, but digit $5$ appears to the left).
\item Any $d128$ with $d\in\{1,2,8\}$ is not a power of two (see Lemma~\ref{lem:len4}).
\end{itemize}
\end{example}

\section{Extension: adding the digit 4}
Consider the alphabet $\{1,2,4,8\}$. Modulo $3$, the digit $4$ is equivalent to $1$, and modulo $6$ its contribution is $4$ (same as $1$), so the invariant of Lemma~\ref{lem:inv} remains valid provided $1$ and $4$ are treated as one class. The phase modulo $6$ now also allows even $n$ with last digit $4$, but among actual powers this yields only $2^2=4$.

\begin{theorem}[Alphabet $\{1,2,4,8\}$]
The only valid powers of two are
\[
2^1=2,\qquad 2^2=4,\qquad 2^3=8,\qquad 2^7=128.
\]
\end{theorem}

\begin{proof}
The invariant and the local carry table still apply. The only new case is $2^2=4$ (last digit $4$ is allowed). The exclusions of length $4$ and of lengths $\ge5$ transfer verbatim.
\end{proof}

\section*{Appendix A: getting endings even faster}
\paragraph{Two digits.} See the table for $2^n\bmod100$ (odd $n$) above; only $12,28,88$ are admissible.
\paragraph{Three digits.} Multiples of $8$ reduce the candidate list to $112,128,288,888$, of which only $112,128,288$ occur along the modulo-$1000$ cycle.

\section*{Appendix B: compact carry table}
The summary table of local rightmost-digit transitions under doubling is given in the “Mini table of carries” section. It shows that only $1\to2$ without carry is admissible, consistent with the stable fragment $11\mapsto22$ and the ending $112\to128$.

\end{document}
